\pdfpageattr {/Group << /S /Transparency /I true /CS /DeviceRGB>>}
\documentclass[10pt, xcolor={usenames, dvipsnames}]{beamer}

\usepackage[french, english]{babel}
\usepackage[utf8]{inputenc}
\usepackage[T1]{fontenc}
\usepackage{upgreek,textgreek}
\usepackage{animate}

\usepackage{pifont}
\newcommand{\cmark}{\ding{51}}
\newcommand{\xmark}{\ding{55}}

\usepackage{amsmath}
\usepackage{amssymb}

\usepackage{graphicx}
\usepackage{hyperref}
\usepackage[style=alphabetic, autocite=inline, firstinits=true, maxnames=2]{biblatex}
\bibliography{bibliographie}
\renewcommand*{\bibfont}{\small}


\usepackage{tabularx, booktabs}
\usepackage{tikz}
\definecolor{light-gray}{gray}{0.65}
\tikzset{fade on/.code={\only<#1>{\color{light-gray}}}}
\tikzset{hide on/.code={\only<#1>{\color{white}}}}
\tikzset{bold on/.code={\only<#1>{\bfseries}}}
\tikzset{
  opinvisible/.style={opacity=0.2},
  visible on/.style={alt={#1{}{opinvisible}}},
  alt/.code args={<#1>#2#3}{%
    \alt<#1>{\pgfkeysalso{#2}}{\pgfkeysalso{#3}} % \pgfkeysalso doesn't change the path
  },
}

\usetikzlibrary{plotmarks,shapes}
\usepackage{colortbl}
\usepackage{ulem}
\usepackage{multicol}
\usepackage[overlay]{textpos}
\usepackage{multirow}
\usepackage{ragged2e}
\usepackage{rotating}
\usepackage{fancybox}
\usepackage{ulem}
\usepackage{overpic}
\usepackage{enumerate}
\usepackage{xfrac}
\usepackage{pgfplots}

\usepackage{transparent}

\usepackage{epstopdf}
\usepackage{epsfig}


\newcommand{\REF}{\textcolor{purple}{REF}}
\newcommand{\GO}[1]{\textcolor{blue}{#1}}
\newcommand{\YES}[1]{\textcolor{green}{#1}}
\newcommand{\NO}[1]{\textcolor{red}{#1}}

\usetheme{metropolis}

\title{Breaking the Scalability Barrier of Causal Broadcast for Large and Dynamic Systems}
\author{Brice N\'edelec, Pascal Molli, and \textbf{Achour Most{\'e}faoui}}
\date{Workshop O'Browser 2018}
\institute{University of Nantes, LS2N}



\begin{document}

\maketitle

\begin{frame}{Introduction}
  Causal broadcast is the core of many distributed applications such as
  distributed social networks, distributed collaborative softwares, or
  distributed data stores.
\end{frame}

\begin{frame}{Static systems: \YES{\cmark}}
  
  Simply broadcast and forward each message exactly once using all outgoing
  links and you have causal broadcast.

  \begin{minipage}{0.32\textwidth}
    \begin{center}
      % \subfloat[Part A][\label{fig:generalsolveA}Process~A broadcasts $a$.]
      
\begin{tikzpicture}[scale=1]
  
  \small
  
  \newcommand\X{210/5pt};
  \newcommand\Y{30pt};
  
  % \draw[->] ( -0.5*\X, 0.5*\Y) -- ( -5+0*\X, 0*\Y);
  % \draw[->] ( -0.5*\X, 0*\Y) -- ( -5+0*\X, 0*\Y);
  % \draw[->] ( -0.5*\X, -0.5*\Y) -- ( -5+0*\X, 0*\Y);  

  \draw[fill=white] (0*\X, 0*\Y) node{\GO{\textbf{A}}} +(-5pt, -5pt) rectangle +(5pt, 5pt);
  \draw[fill=white] (1*\X, -1*\Y) node{\textbf{B}} +(-5pt, -5pt) rectangle +(5pt, 5pt);
  \draw[fill=white] (2*\X,  0*\Y) node{\textbf{C}} +(-5pt, -5pt) rectangle +(5pt, 5pt);

  \draw[->](5+0*\X, 0*\Y) -- node[sloped, above]{\GO{$a$}} (-5+1*\X, -1*\Y); %% A->B
  \draw[<-](5+0*\X, -5+0*\Y) -- (-5+1*\X, -5-1*\Y); %% A<-B

  \draw[->](5+0*\X, 5+0*\Y) node[above right]{\GO{$\,\,a$}} --  (-5+2*\X, 5+0*\Y); % A->C
  \draw[<-](5+0*\X,  1.25+ 0*\Y) -- (-5+2*\X,  1.25+ 0*\Y); % A<-C
 
  \draw[<-](5+1*\X, -1*\Y) -- (-5+2*\X, 0*\Y); %% B<-C
  \draw[->](5+1*\X, -5-1*\Y) -- (-5+2*\X, -5+0*\Y); %% B->C

  % \draw[->](5+2*\X, 0*\Y) -- ( 2.5*\X, 0*\Y);
\end{tikzpicture}
    \end{center}
  \end{minipage}
  \begin{minipage}{0.32\textwidth}
    \begin{center}
      % \subfloat[Part B][\label{fig:generalsolveB}Process~B receives and 
      % delivers $a$. Process~B forwards $a$ using its FIFO links.]
      
\begin{tikzpicture}[scale=1]
  
  \small
  
  \newcommand\X{210/5pt};
  \newcommand\Y{30pt};
  
  % \draw[->] ( -0.5*\X, 0.5*\Y) -- ( -5+0*\X, 0*\Y);
  % \draw[->] ( -0.5*\X, 0*\Y) -- ( -5+0*\X, 0*\Y);
  % \draw[->] ( -0.5*\X, -0.5*\Y) -- ( -5+0*\X, 0*\Y);  

  \draw[fill=white] (0*\X, 0*\Y) node{\textbf{A}} +(-5pt, -5pt) rectangle +(5pt, 5pt);
  \draw[fill=white] (1*\X, -1*\Y) node{\GO{\textbf{B}}} +(-5pt, -5pt) rectangle +(5pt, 5pt);
  \draw[fill=white] (2*\X,  0*\Y) node{\textbf{C}} +(-5pt, -5pt) rectangle +(5pt, 5pt);
  \draw (1*\X, 5-1*\Y) node[above]{$a$};

  \draw[->](5+0*\X, 0*\Y) -- (-5+1*\X, -1*\Y); %% A->B
  \draw[<-](5+0*\X, -5+0*\Y) -- node[sloped, below right]{\GO{$\,\,a$}} (-5+1*\X, -5-1*\Y); %% A<-B

  \draw[->](5+0*\X, 5+0*\Y) -- node[above left]{$a$} (-5+2*\X, 5+0*\Y); % A->C
  \draw[<-](5+0*\X,  1.25+ 0*\Y) -- (-5+2*\X,  1.25+ 0*\Y); % A<-C
 
  \draw[<-](5+1*\X, -1*\Y) -- (-5+2*\X, 0*\Y); %% B<-C
  \draw[->](5+1*\X, -5-1*\Y) --  node[sloped, below left]{\GO{$a\,\,$}} (-5+2*\X, -5+0*\Y); %% B->C

  % \draw[->](5+2*\X, 0*\Y) -- ( 2.5*\X, 0*\Y);
\end{tikzpicture}
    \end{center}
  \end{minipage}
  \begin{minipage}{0.32\textwidth}
    \begin{center}
      % \subfloat[Part C][\label{fig:generalsolveC}Process~B broadcasts $b$.]
      
\begin{tikzpicture}[scale=1]
  
  \small
  
  \newcommand\X{210/5pt};
  \newcommand\Y{30pt};
  
  % \draw[->] ( -0.5*\X, 0.5*\Y) -- ( -5+0*\X, 0*\Y);
  % \draw[->] ( -0.5*\X, 0*\Y) -- ( -5+0*\X, 0*\Y);
  % \draw[->] ( -0.5*\X, -0.5*\Y) -- ( -5+0*\X, 0*\Y);  

  \draw[fill=white] (0*\X, 0*\Y) node{\textbf{A}} +(-5pt, -5pt) rectangle +(5pt, 5pt);
  \draw[fill=white] (1*\X, -1*\Y) node{\GO{\textbf{B}}} +(-5pt, -5pt) rectangle +(5pt, 5pt);
  \draw[fill=white] (2*\X,  0*\Y) node{\textbf{C}} +(-5pt, -5pt) rectangle +(5pt, 5pt);
%  \draw (1*\X, 5-1*\Y) node[above]{$a$};

  \draw[->](5+0*\X, 0*\Y) -- (-5+1*\X, -1*\Y); %% A->B
  \draw[<-](5+0*\X, -5+0*\Y) -- node[sloped, below]{$a\,\,\,\,\,\,$ \GO{$b$}}
  (-5+1*\X, -5-1*\Y); %% A<-B

  \draw[->](5+0*\X, 5+0*\Y) -- node[above]{$a$} (-5+2*\X, 5+0*\Y); % A->C
  \draw[<-](5+0*\X,  1.25+ 0*\Y) -- (-5+2*\X,  1.25+ 0*\Y); % A<-C
 
  \draw[<-](5+1*\X, -1*\Y) -- (-5+2*\X, 0*\Y); %% B<-C
  \draw[->](5+1*\X, -5-1*\Y) --  node[sloped, below]{$\GO{b}$$\,\,\,\,\,\,a$} (-5+2*\X, -5+0*\Y); %% B->C

  % \draw[->](5+2*\X, 0*\Y) -- ( 2.5*\X, 0*\Y);
\end{tikzpicture}
    \end{center}
  \end{minipage}
  
  \begin{center}
  \begin{minipage}{0.35\textwidth}
    \begin{center}
    % \subfloat[Part D][\label{fig:generalsolveD}Process~A receives and 
    % delivers $b$. Process~A forwards $b$ using its FIFO links.]
      
\begin{tikzpicture}[scale=1]
  
  \small
  
  \newcommand\X{210/5pt};
  \newcommand\Y{30pt};
  
  % \draw[->] ( -0.5*\X, 0.5*\Y) -- ( -5+0*\X, 0*\Y);
  % \draw[->] ( -0.5*\X, 0*\Y) -- ( -5+0*\X, 0*\Y);
  % \draw[->] ( -0.5*\X, -0.5*\Y) -- ( -5+0*\X, 0*\Y);  

  \draw[fill=white] (0*\X, 0*\Y) node{\GO{\textbf{A}}} +(-5pt, -5pt) rectangle +(5pt, 5pt);
  \draw (-5 + 0*\X, 0*\Y) node[left]{$b$};
  \draw[fill=white] (1*\X, -1*\Y) node{\textbf{B}} +(-5pt, -5pt) rectangle +(5pt, 5pt);
  \draw[fill=white] (2*\X,  0*\Y) node{\textbf{C}} +(-5pt, -5pt) rectangle +(5pt, 5pt);
%  \draw (1*\X, 5-1*\Y) node[above]{$a$};

  \draw[->](5+0*\X, 0*\Y) --
  node[sloped, above left]{$\GO{b}\,\,\,$}
  (-5+1*\X, -1*\Y); %% A->B
  \draw[<-](5+0*\X, -5+0*\Y) -- (-5+1*\X, -5-1*\Y); %% A<-B

  \draw[->](5+0*\X, 5+0*\Y) node[above right]{$\GO{b}$}-- node[above right]{$a$} (-5+2*\X, 5+0*\Y); % A->C
  \draw[<-](5+0*\X,  1.25+ 0*\Y) -- (-5+2*\X,  1.25+ 0*\Y); % A<-C
 
  \draw[<-](5+1*\X, -1*\Y) -- (-5+2*\X, 0*\Y); %% B<-C
  \draw[->](5+1*\X, -5-1*\Y) --  node[sloped, below]{$b\,\,\,\,\,\,a$} (-5+2*\X, -5+0*\Y); %% B->C

  % \draw[->](5+2*\X, 0*\Y) -- ( 2.5*\X, 0*\Y);
\end{tikzpicture}
    \end{center}
  \end{minipage}
  \begin{minipage}{0.35\textwidth}
    \begin{center}
    % \hspace{40pt}
    % \subfloat[Part E][\label{fig:generalsolveE}Process~C cannot receive $b$
    % without having received $a$ beforehand.]
      
\begin{tikzpicture}[scale=1]
  
  \small
  
  \newcommand\X{210/5pt};
  \newcommand\Y{30pt};
  
  % \draw[->] ( -0.5*\X, 0.5*\Y) -- ( -5+0*\X, 0*\Y);
  % \draw[->] ( -0.5*\X, 0*\Y) -- ( -5+0*\X, 0*\Y);
  % \draw[->] ( -0.5*\X, -0.5*\Y) -- ( -5+0*\X, 0*\Y);  

  \draw[fill=white] (0*\X, 0*\Y) node{\textbf{A}} +(-5pt, -5pt) rectangle +(5pt, 5pt);
  \draw[fill=white] (1*\X, -1*\Y) node{\textbf{B}} +(-5pt, -5pt) rectangle +(5pt, 5pt);
  \draw[fill=white] (2*\X,  0*\Y) node{\textbf{C}} +(-5pt, -5pt) rectangle +(5pt, 5pt);

  \draw[->](5+0*\X, 0*\Y) -- (-5+1*\X, -1*\Y); %% A->B
  \draw[<-](5+0*\X, -5+0*\Y) -- (-5+1*\X, -5-1*\Y); %% A<-B

  \draw[->](5+0*\X, 5+0*\Y) -- (-5+2*\X, 5+0*\Y) node[above left]{$\GO{\pmb{b\,a}}\,\,$}; % A->C
  \draw[<-](5+0*\X,  1.25+ 0*\Y) -- (-5+2*\X,  1.25+ 0*\Y); % A<-C
 
  \draw[<-](5+1*\X, -1*\Y) -- (-5+2*\X, 0*\Y); %% B<-C
  \draw[->](5+1*\X, -5-1*\Y) -- node[sloped, below right]{$\,\,\,\,\,\GO{\pmb{b\,a}}$} (-5+2*\X, -5+0*\Y); %% B->C

  % \draw[->](5+2*\X, 0*\Y) -- ( 2.5*\X, 0*\Y);
\end{tikzpicture}
    % \caption{\label{fig:generalsolve}Causal broadcast~\cite{friedman2004causal}
    %   ensures causal order.}
    \end{center}
  \end{minipage}
\end{center}

Propagation ensures causal order by design. It is very efficient for message
overhead remains constant $O(1)$.

\end{frame}

\begin{frame}{Large scale static systems: \YES{\cmark}}

  Each process has a partial view much smaller than the actual system size and
  efficiently propagates messages using gossiping.

  \begin{center}
    
\begin{tikzpicture}[scale=0.65]

  \small
  
  \newcommand\X{210/5pt};
  \newcommand\Y{30pt};
  
  \draw[->] ( -1*\X, 1*\Y) -- ( -5+0*\X, 0*\Y);
  \draw[->] ( -1*\X, 0*\Y) node[anchor=east, align=center]{Rest\\of the\\network} -- ( -5+0*\X, 0*\Y);
  \draw[->] ( -1*\X, -1*\Y) -- ( -5+0*\X, 0*\Y);

  \draw[<-] ( 5*\X, 1*\Y) -- ( 5+4*\X, 0*\Y);
  \draw[<-] ( 5*\X, 0*\Y) node[anchor=west, align=center]{Rest\\of the\\network}-- ( 5+4*\X, 0*\Y);
  \draw[<-] ( 5*\X, -1*\Y) -- ( 5+4*\X, 0*\Y);


  \draw[fill=white] (0*\X, 0*\Y) node{\textbf{A}} +(-5pt, -5pt) rectangle +(5pt, 5pt);

  \draw[fill=white] (1*\X, -1*\Y) node{\textbf{B}} +(-5pt, -5pt) rectangle +(5pt, 5pt);

  \draw[fill=white] (2*\X, -2*\Y) node{\textbf{E}} +(-5pt, -5pt) rectangle +(5pt, 5pt);
  \draw[fill=white] (2*\X,  0*\Y) node{\textbf{D}} +(-5pt, -5pt) rectangle +(5pt, 5pt);
  \draw[fill=white] (2*\X,  2*\Y) node{\textbf{C}} +(-5pt, -5pt) rectangle +(5pt, 5pt);

  \draw[fill=white] (4*\X, 0*\Y) node{\textbf{F}} +(-5pt, -5pt) rectangle +(5pt, 5pt);

  \draw[->](5+0*\X, 0*\Y) -- node[sloped, above]{\GO{\pmb{$a'\,a$}}} (-5+2*\X,  2*\Y); %% A->E
  \draw[->](5+0*\X, 0*\Y) -- node[sloped, above]{\GO{\pmb{$a'\,a$}}} (-5+1*\X, -1*\Y); %% A->B

  \draw[->](5+1*\X, -1*\Y) -- node[sloped, above]{$\pmb{a'}\,b\,\pmb{a}$} (-5+2*\X, 0*\Y); %% B->D
  \draw[->](5+1*\X, -1*\Y) -- node[sloped, above]{$\pmb{a'}\,b\,\pmb{a}$} (-5+2*\X, -2*\Y); %% B->C

  \draw[->](5+2*\X,  2*\Y) -- node[sloped, above]{$c\,\GO{\pmb{a'\,a}}$} (-5+4*\X, 0*\Y); %% E->F
  \draw[->](5+2*\X,  0*\Y) -- node[sloped, above]{$\GO{\pmb{a'}}\,b\,\GO{\pmb{a}}\,d$} (-5+4*\X, 0*\Y); %% D->F
  \draw[->](5+2*\X, -2*\Y) -- node[sloped, above]{$\GO{\pmb{a'}}\,b\,\,e'\,e\,\GO{\pmb{a}}$} (-5+4*\X, 0*\Y); %% C->F

  


\end{tikzpicture}
  \end{center}

  Gossiping is already mandatory for large scale systems. It is not an
  additional overhead of causal broadcast.

\end{frame}



\begin{frame}{But dynamic systems: \NO{\xmark}}

  New links may act as shortcut for new messages impairing causal order.

\begin{minipage}{0.32\textwidth}
    \begin{center}
      % \subfloat[Part A][\label{fig:generalsolveA}Process~A broadcasts $a$.]
      
\begin{tikzpicture}[scale=0.75]
  
  \small
  
  \newcommand\X{210/5pt};
  \newcommand\Y{30pt};
  
  \draw[->] ( -0.5*\X, 0.5*\Y) -- ( -5+0*\X, 0*\Y);
  \draw[->] ( -0.5*\X, 0*\Y) -- ( -5+0*\X, 0*\Y);
  \draw[->] ( -0.5*\X, -0.5*\Y) -- ( -5+0*\X, 0*\Y);  

  \draw[fill=white] (0*\X, 0*\Y) node{\GO{\textbf{A}}} +(-5pt, -5pt) rectangle +(5pt, 5pt);
  \draw[fill=white] (1*\X, -1*\Y) node{\textbf{B}} +(-5pt, -5pt) rectangle +(5pt, 5pt);
  \draw[fill=white] (2*\X,  0*\Y) node{\textbf{D}} +(-5pt, -5pt) rectangle +(5pt, 5pt);

  \draw[->](5+0*\X, 0*\Y) -- node[sloped, above left]{\GO{$a$}} (-5+1*\X, -1*\Y); %% A->B
  \draw[->](5+1*\X, -1*\Y) -- (-5+2*\X, 0*\Y); %% B->D

  \draw[->](5+2*\X, 0*\Y) -- ( 2.5*\X, 0*\Y);
\end{tikzpicture}
    \end{center}
  \end{minipage}
  \begin{minipage}{0.32\textwidth}
    \begin{center}
      % \subfloat[Part B][\label{fig:generalsolveB}Process~B receives and 
      % delivers $a$. Process~B forwards $a$ using its FIFO links.]
      
\begin{tikzpicture}[scale=0.75]
  
  \small
  
  \newcommand\X{210/5pt};
  \newcommand\Y{30pt};
  
  \draw[->] ( -0.5*\X, 0.5*\Y) -- ( -5+0*\X, 0*\Y);
  \draw[->] ( -0.5*\X, 0*\Y) -- ( -5+0*\X, 0*\Y);
  \draw[->] ( -0.5*\X, -0.5*\Y) -- ( -5+0*\X, 0*\Y);  

  \draw[fill=white] (0*\X, 0*\Y) node{\textbf{A}} +(-5pt, -5pt) rectangle +(5pt, 5pt);
  \draw[fill=white] (1*\X, -1*\Y) node{\textbf{B}} +(-5pt, -5pt) rectangle +(5pt, 5pt);
  \draw[fill=white] (2*\X,  0*\Y) node{\textbf{D}} +(-5pt, -5pt) rectangle +(5pt, 5pt);

  \draw[->](5+0*\X, 0*\Y) -- node[sloped, above right]{$a$} (-5+1*\X, -1*\Y); %% A->B
  \draw[->](5+1*\X, -1*\Y) -- (-5+2*\X, 0*\Y); %% B->D

  \draw[->, color=blue] (5+0*\X, 0*\Y) -- (-5+2*\X, 0*\Y);

  \draw[->](5+2*\X, 0*\Y) -- ( 2.5*\X, 0*\Y);
\end{tikzpicture}
    \end{center}
  \end{minipage}
  \begin{minipage}{0.32\textwidth}
    \begin{center}
      % \subfloat[Part C][\label{fig:generalsolveC}Process~B broadcasts $b$.]
      
\begin{tikzpicture}[scale=0.75]
  
  \small
  
  \newcommand\X{210/5pt};
  \newcommand\Y{30pt};
  
  \draw[->] ( -0.5*\X, 0.5*\Y) -- ( -5+0*\X, 0*\Y);
  \draw[->] ( -0.5*\X, 0*\Y) -- ( -5+0*\X, 0*\Y);
  \draw[->] ( -0.5*\X, -0.5*\Y) -- ( -5+0*\X, 0*\Y);  

  \draw[fill=white] (0*\X, 0*\Y) node{\GO{\textbf{A}}} +(-5pt, -5pt) rectangle +(5pt, 5pt);
  \draw[fill=white] (1*\X, -1*\Y) node{\textbf{B}} +(-5pt, -5pt) rectangle +(5pt, 5pt);
  \draw (1*\X, 5-1*\Y) node[anchor=south]{$a$};
  \draw[fill=white] (2*\X,  0*\Y) node{\textbf{D}} +(-5pt, -5pt) rectangle +(5pt, 5pt);

  \draw[->](5+0*\X, 0*\Y) -- node[sloped, above]{\GO{$a'$}} (-5+1*\X, -1*\Y); %% A->B
  \draw[->](5+1*\X, -1*\Y) -- (-5+2*\X, 0*\Y); %% B->D
  
  \draw[->] (5+0*\X, 0*\Y) -- node[anchor=south]{\NO{$a'$}} (-5+2*\X, 0*\Y); %% A->B

  \draw[->](5+2*\X, 0*\Y) -- ( 2.5*\X, 0*\Y);
\end{tikzpicture}
    \end{center}
  \end{minipage}
  
  \begin{center}
  \begin{minipage}{0.35\textwidth}
    \begin{center}
    % \subfloat[Part D][\label{fig:generalsolveD}Process~A receives and 
    % delivers $b$. Process~A forwards $b$ using its FIFO links.]
      
\begin{tikzpicture}[scale=0.75]
  
  \small
  
  \newcommand\X{210/5pt};
  \newcommand\Y{30pt};
  
  \draw[->] ( -0.5*\X, 0.5*\Y) -- ( -5+0*\X, 0*\Y);
  \draw[->] ( -0.5*\X, 0*\Y) -- ( -5+0*\X, 0*\Y);
  \draw[->] ( -0.5*\X, -0.5*\Y) -- ( -5+0*\X, 0*\Y);  

  \draw[fill=white] (0*\X, 0*\Y) node{\textbf{A}} +(-5pt, -5pt) rectangle +(5pt, 5pt);
  \draw[fill=white] (1*\X, -1*\Y) node{\textbf{B}} +(-5pt, -5pt) rectangle +(5pt, 5pt);
  \draw (1*\X, 5-1*\Y) node[anchor=south]{$a'$};
  \draw[fill=white] (2*\X,  0*\Y) node{\NO{\textbf{D}}} +(-5pt, -5pt) rectangle +(5pt, 5pt);
  \draw (2*\X, 5-0*\Y) node[anchor=south]{$\NO{\pmb{a'}}$};

  \draw[->](5+0*\X, 0*\Y) --  (-5+1*\X, -1*\Y); %% A->B
  \draw[->](5+1*\X, -1*\Y) -- node[sloped, above]{$b\,\pmb{a}$} (-5+2*\X, 0*\Y); %% B->D
  
  \draw[->] (5+0*\X, 0*\Y) -- (-5+2*\X, 0*\Y); %% A->B

  \draw[->](5+2*\X, 0*\Y) -- ( 2.5*\X, 0*\Y);

\end{tikzpicture}
    \end{center}
  \end{minipage}
  \begin{minipage}{0.35\textwidth}
    \begin{center}
    % \hspace{40pt}
    % \subfloat[Part E][\label{fig:generalsolveE}Process~C cannot receive $b$
    % without having received $a$ beforehand.]
      
\begin{tikzpicture}[scale=0.75]
  
  \small
  
  \newcommand\X{210/5pt};
  \newcommand\Y{30pt};
  
  \draw[->] ( -0.5*\X, 0.5*\Y) -- ( -5+0*\X, 0*\Y);
  \draw[->] ( -0.5*\X, 0*\Y) -- ( -5+0*\X, 0*\Y);
  \draw[->] ( -0.5*\X, -0.5*\Y) -- ( -5+0*\X, 0*\Y);  

  \draw[fill=white] (0*\X, 0*\Y) node{\textbf{A}} +(-5pt, -5pt) rectangle +(5pt, 5pt);
  \draw[fill=white] (1*\X, -1*\Y) node{\textbf{B}} +(-5pt, -5pt) rectangle +(5pt, 5pt);
  \draw[fill=white] (2*\X,  0*\Y) node{\textbf{C}} +(-5pt, -5pt) rectangle +(5pt, 5pt);
  \draw (2*\X, 5-0*\Y) node[anchor=south]{$b$};

  \draw[->](5+0*\X, 0*\Y) --  (-5+1*\X, -1*\Y); %% A->B
  \draw[->](5+1*\X, -1*\Y) -- node[sloped, above]{$a'$} (-5+2*\X, 0*\Y); %% B->D
  
  \draw[->] (5+0*\X, 0*\Y) -- (-5+2*\X, 0*\Y); %% A->B

  \draw[->](5+2*\X, 0*\Y) -- node[above right]{\NO{$\pmb{a\,a'}$}} ( 2.5*\X, 0*\Y);

\end{tikzpicture}
    % \caption{\label{fig:generalsolve}Causal broadcast~\cite{friedman2004causal}
    %   ensures causal order.}
    \end{center}
  \end{minipage}
  \end{center}  

  Not only Process~D fails to deliver messages in causal order but propagates the
  mistake to the rest of processes.

\end{frame}


\begin{frame}[standout]
  How do we keep the complexity of causal broadcast for large scale static
  systems in dynamic settings where processes can join, leave, or
  self-reconfigure at any time?
\end{frame}

\begin{frame}{Proposal: PC-broadcast}  

  PC-broadcast stands for Preventive Causal broadcast. It prevents causal order
  violations by forbidding the usage of new links until proven safe. Each
  process uses all and only safe links for causal broadcast.
  
  
  \begin{definition}[\label{def:safe}Safe link]
    A link from Process~A to Process~B is safe if and only if Process~B received
    or will receive all messages delivered by Process~A before receiving any
    message that Process~A will deliver:
    $safe_{AB} \equiv \forall m,\, m',\, d_A(m) \rightarrow s_{AB}(m') \implies
    r_B(m) \rightarrow r_{BA}(m')$
  \end{definition}

  Links start unsafe. The challenge is to make links safe using local knowledge
  only. Sending all delivered messages since the beginning makes safe links but
  is over-expansive: \NO{\xmark}. We will use few lightweight control messages
  that follow causal order: \YES{\cmark}.

\end{frame}


\begin{frame}{Control messages and buffering make safe links}
  
  \begin{enumerate}
  \item Sending a control message using safe link to the other process
    $\rightarrow$ will acknowledge the delivery of most of past messages
  \item Buffer all delivered messages $\rightarrow$ saves concurrent messages to
    safe link establishment
  \item Upon acknowledgment of the control message, send the buffer of
    concurrent messages
  \end{enumerate}


\begin{minipage}{0.32\textwidth}
    \begin{center}
      % \subfloat[Part A][\label{fig:generalsolveA}Process~A broadcasts $a$.]
      
\begin{tikzpicture}[scale=0.75]
  
  \small
  
  \newcommand\X{210/5pt};
  \newcommand\Y{30pt};
  
  \draw[->] ( -0.5*\X, 0.5*\Y) -- ( -5+0*\X, 0*\Y);
  \draw[->] ( -0.5*\X, 0*\Y) -- ( -5+0*\X, 0*\Y);
  \draw[->] ( -0.5*\X, -0.5*\Y) -- ( -5+0*\X, 0*\Y);  

  \draw[fill=white] (0*\X, 0*\Y) node{\textbf{A}} +(-5pt, -5pt) rectangle +(5pt, 5pt);
  \draw[fill=white] (1*\X, -1*\Y) node{\textbf{B}} +(-5pt, -5pt) rectangle +(5pt, 5pt);
  \draw[fill=white] (2*\X,  0*\Y) node{\textbf{D}} +(-5pt, -5pt) rectangle +(5pt, 5pt);

  \draw[->](5+0*\X, 0*\Y) -- node[sloped, above]{$a$} (-5+1*\X, -1*\Y); %% A->B
  \draw[->](5+1*\X, -1*\Y) -- (-5+2*\X, 0*\Y); %% B->D

  \draw[->](5+2*\X, 0*\Y) -- ( 2.5*\X, 0*\Y);
\end{tikzpicture}
    \end{center}
  \end{minipage}
  \begin{minipage}{0.32\textwidth}
    \begin{center}
      % \subfloat[Part B][\label{fig:generalsolveB}Process~B receives and 
      % delivers $a$. Process~B forwards $a$ using its FIFO links.]
      
\begin{tikzpicture}[scale=0.75]
  
  \small
  
  \newcommand\X{210/5pt};
  \newcommand\Y{30pt};
  
  \draw[->] ( -0.5*\X, 0.5*\Y) -- ( -5+0*\X, 0*\Y);
  \draw[->] ( -0.5*\X, 0*\Y) -- ( -5+0*\X, 0*\Y);
  \draw[->] ( -0.5*\X, -0.5*\Y) -- ( -5+0*\X, 0*\Y);  

  \draw[fill=white] (0*\X, 0*\Y) node{\textbf{A}} +(-5pt, -5pt) rectangle +(5pt, 5pt);
  \draw[fill=white] (1*\X, -1*\Y) node{\textbf{B}} +(-5pt, -5pt) rectangle +(5pt, 5pt);
  \draw (1*\X, 5-1*\Y) node[anchor=south]{$a$};
  \draw[fill=white] (2*\X,  0*\Y) node{\textbf{C}} +(-5pt, -5pt) rectangle +(5pt, 5pt);

  \draw[->](5+0*\X, 0*\Y) -- node[sloped, above]{\GO{$\pi$}} (-5+1*\X, -1*\Y); %% A->B
  \draw[->](5+1*\X, -1*\Y) -- (-5+2*\X, 0*\Y); %% B->D
  
  \draw[->, dashed, color=blue] (5+0*\X, 0*\Y) node[anchor=south west]{\GO{$[\,]$}} -- (-5+2*\X, 0*\Y); %% A->B

  \draw[->](5+2*\X, 0*\Y) -- ( 2.5*\X, 0*\Y);
\end{tikzpicture}
    \end{center}
  \end{minipage}
  \begin{minipage}{0.32\textwidth}
    \begin{center}
      % \subfloat[Part C][\label{fig:generalsolveC}Process~B broadcasts $b$.]
      
\begin{tikzpicture}[scale=0.75]
  
  \small
  
  \newcommand\X{210/5pt};
  \newcommand\Y{30pt};
  
  \draw[->] ( -0.5*\X, 0.5*\Y) -- ( -5+0*\X, 0*\Y);
  \draw[->] ( -0.5*\X, 0*\Y) -- ( -5+0*\X, 0*\Y);
  \draw[->] ( -0.5*\X, -0.5*\Y) -- ( -5+0*\X, 0*\Y);  

  \draw[fill=white] (0*\X, 0*\Y) node{\textbf{A}} +(-5pt, -5pt) rectangle +(5pt, 5pt);
  \draw[fill=white] (1*\X, -1*\Y) node{\textbf{B}} +(-5pt, -5pt) rectangle +(5pt, 5pt);
%  \draw (1*\X, 5-1*\Y) node[anchor=south]{$a$};
  \draw[fill=white] (2*\X,  0*\Y) node{\textbf{D}} +(-5pt, -5pt) rectangle +(5pt, 5pt);

  \draw[->](5+0*\X, 0*\Y) -- node[sloped, above]{$a'$} (-5+1*\X, -1*\Y); %% A->B
  \draw[->](5+1*\X, -1*\Y) -- node[sloped, above]{$\GO{\pi}\,b\,a$} (-5+2*\X, 0*\Y); %% B->D
  
  \draw[->, dashed] (5+0*\X, 0*\Y) node[anchor=south west]{$[\GO{a'}]$} --  (-5+2*\X, 0*\Y); %% A->B

  \draw[->](5+2*\X, 0*\Y) -- ( 2.5*\X, 0*\Y);
\end{tikzpicture}
    \end{center}
  \end{minipage}
  
  \begin{center}
  \begin{minipage}{0.35\textwidth}
    \begin{center}
    % \subfloat[Part D][\label{fig:generalsolveD}Process~A receives and 
    % delivers $b$. Process~A forwards $b$ using its FIFO links.]
      
\begin{tikzpicture}[scale=0.75]
  
  \small
  
  \newcommand\X{210/5pt};
  \newcommand\Y{30pt};
  
  \draw[->] ( -0.5*\X, 0.5*\Y) -- ( -5+0*\X, 0*\Y);
  \draw[->] ( -0.5*\X, 0*\Y) -- ( -5+0*\X, 0*\Y);
  \draw[->] ( -0.5*\X, -0.5*\Y) -- ( -5+0*\X, 0*\Y);  

  \draw[fill=white] (0*\X, 0*\Y) node{\textbf{A}} +(-5pt, -5pt) rectangle +(5pt, 5pt);
  \draw[fill=white] (1*\X, -1*\Y) node{\textbf{B}} +(-5pt, -5pt) rectangle +(5pt, 5pt);
  \draw (1*\X, 5-1*\Y) node[anchor=south]{$a'$};
  \draw[fill=white] (2*\X,  0*\Y) node{\textbf{C}} +(-5pt, -5pt) rectangle +(5pt, 5pt);
  \draw (2*\X, 5-0*\Y) node[anchor=south]{$\pi$};

  \draw[->](5+0*\X, 0*\Y) -- (-5+1*\X, -1*\Y); %% A->B
  \draw[->](5+1*\X, -1*\Y) -- (-5+2*\X, 0*\Y); %% B->D
  
  \draw[->, dashed] (5+0*\X, 0*\Y) node[anchor=south west]{$[a']$} --  (-5+2*\X, 0*\Y); %% A->B

  \draw[->, decorate, decoration={snake, amplitude=0.3mm}, color=blue](-5+2*\X, 5+0*\Y)
  to[out=180-25, in=25] node[sloped, above left]{\GO{$\rho$}}(0.65*\X, 10+0*\Y); 

  \draw[->](5+2*\X, 0*\Y) -- node[anchor=south]{$b\,a$}( 2.5*\X, 0*\Y);
\end{tikzpicture}
    \end{center}
  \end{minipage}
  \begin{minipage}{0.35\textwidth}
    \begin{center}
    % \hspace{40pt}
    % \subfloat[Part E][\label{fig:generalsolveE}Process~C cannot receive $b$
    % without having received $a$ beforehand.]
      
\begin{tikzpicture}[scale=0.75]
  
  \small
  
  \newcommand\X{210/5pt};
  \newcommand\Y{30pt};
  
  \draw[->] ( -0.5*\X, 0.5*\Y) -- ( -5+0*\X, 0*\Y);
  \draw[->] ( -0.5*\X, 0*\Y) -- ( -5+0*\X, 0*\Y);
  \draw[->] ( -0.5*\X, -0.5*\Y) -- ( -5+0*\X, 0*\Y);  

  \draw[fill=white] (0*\X, 0*\Y) node{\textbf{A}} +(-5pt, -5pt) rectangle +(5pt, 5pt);
  \draw (0*\X, 5-0*\Y) node[anchor=south]{$\rho$};
  \draw[fill=white] (1*\X, -1*\Y) node{\textbf{B}} +(-5pt, -5pt) rectangle +(5pt, 5pt);
  \draw[fill=white] (2*\X,  0*\Y) node{\textbf{D}} +(-5pt, -5pt) rectangle +(5pt, 5pt);

  \draw[->](5+0*\X, 0*\Y) -- (-5+1*\X, -1*\Y); %% A->B
  \draw[->](5+1*\X, -1*\Y) -- node[sloped, above]{$a'$} (-5+2*\X, 0*\Y); %% B->D
  
  \draw[->, color=blue] (5+0*\X, 0*\Y)  -- node[anchor=south]{\GO{$a'$}} (-5+2*\X, 0*\Y); %% A->B

  \draw[->](5+2*\X, 0*\Y) -- node[anchor=south west]{$b$}( 2.5*\X, 0*\Y);
\end{tikzpicture}
    % \caption{\label{fig:generalsolve}Causal broadcast~\cite{friedman2004causal}
    %   ensures causal order.}
    \end{center}
  \end{minipage}
  \end{center}    

\end{frame}


\begin{frame}{Network conditions may lead to unbounded space consumption}
\end{frame}

\begin{frame}{Handling failures}
\end{frame}

\begin{frame}{Experimentation: small negative impact on the overlay}
\end{frame}

\begin{frame}{Conclusion}
\end{frame}

\begin{frame}{To be continued\ldots}
  Removing the last monotonic upper bound on space complexity
\end{frame}

\end{document}
